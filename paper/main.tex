\documentclass[conference]{IEEEtran}
\IEEEoverridecommandlockouts

\usepackage[backend=biber]{biblatex}
\addbibresource{references.bib}
\usepackage{amsmath,amssymb,amsfonts}
\usepackage{algorithmic}
\usepackage{graphicx}
\usepackage{textcomp}
\usepackage{xcolor}
\usepackage{listings}
\usepackage{booktabs}
\usepackage{hyperref}
\usepackage{cleveref}

% YAML listing style
\lstdefinestyle{yaml}{
  basicstyle=\ttfamily\footnotesize,
  breaklines=true,
  frame=single,
  numbers=left,
  numberstyle=\tiny\color{gray},
  keywordstyle=\color{blue},
  stringstyle=\color{orange},
  commentstyle=\color{gray},
  showstringspaces=false,
  tabsize=2
}

\def\BibTeX{{\rm B\kern-.05em{\sc i\kern-.025em b}\kern-.08em
    T\kern-.1667em\lower.7ex\hbox{E}\kern-.125emX}}

\begin{document}

\title{TMDL: A Description Language for Virtual Teammates in Hybrid Human-AI Teams}

\author{
    \IEEEauthorblockN{Pedro A. Pernías Peco}
    \IEEEauthorblockA{
        Departamento de Lenguajes y Sistemas Informáticos\\
        Universidad de Alicante\\
        Alicante, España\\
        p.pernias@ua.es
    }
    \and
    \IEEEauthorblockN{M. Pilar Escobar Esteban}
    \IEEEauthorblockA{
        Departamento de Lenguajes y Sistemas Informáticos\\
        Universidad de Alicante\\
        Alicante, España\\
        pilar.escobar@ua.es
    }
}

\maketitle

\begin{abstract}
The integration of Large Language Models (LLMs) into collaborative work environments presents a fundamental design choice: should AI function as a \textit{tool} for ad-hoc queries or as a \textit{teammate} with defined identity and role? This paper introduces TMDL (TeamMate Description Language), a YAML-based specification language for describing virtual teammates in hybrid human-AI teams. Building on ADL (Assistant Description Language), the COHUMAIN framework for collective human-machine intelligence, and organizational behavior research, TMDL provides structured specifications for identity, role, collaboration patterns, and project knowledge. Key contributions include: (1) a contribution-centered collaboration model that frames AI teammates around how they add value to the group; (2) taxonomies for contribution styles and cognitive patterns derived from established team composition research; and (3) explicit boundary specifications that position AI as a configured partner operating under human direction. We derive six research hypotheses comparing the teammate paradigm against tool-based AI use and propose a quasi-experimental validation approach. The specification is released as open-source at \url{https://github.com/ppernias/tmdl}.
\end{abstract}

\begin{IEEEkeywords}
Large Language Models, Human-AI Collaboration, Agent Description Languages, Virtual Teams, Prompt Engineering
\end{IEEEkeywords}

% ==============================================================================
\section{Introduction}
% ==============================================================================

The rapid advancement of Large Language Models (LLMs) has opened unprecedented opportunities for human-AI collaboration in professional and academic settings. However, deploying LLMs as effective team members rather than mere tools presents unique challenges that go beyond traditional prompt engineering approaches.

When humans collaborate in teams, they bring not only their skills but also their personalities, communication styles, and understanding of team dynamics. Research on AI as teammates has identified significant open challenges in this integration---including role distribution, trust dynamics, and coordination patterns---that fundamentally differ from traditional tool-based human-computer interaction \cite{seeber2020machines}.

Consider a project team where an AI agent is expected to serve as a research analyst. Beyond answering questions, the agent must understand its role boundaries, know when to defer to others, maintain consistent communication patterns, and adapt its behavior to project phases and team needs. Encoding all these requirements in unstructured natural language prompts leads to several problems: ambiguity in interpretation, prompt drift as specifications evolve, lack of validation mechanisms, and poor reproducibility across deployments.

This paper introduces TMDL (TeamMate Description Language), an evolution of ADL (Assistant Description Language) \cite{pernias2025adl} specifically designed for team collaboration contexts. TMDL addresses these challenges by providing:

\begin{itemize}
    \item A \textbf{formal schema} for describing AI teammates with consistent structure and validation
    \item \textbf{Separation of concerns} between identity, role, collaboration patterns, and project knowledge
    \item \textbf{Theoretically-grounded taxonomies} for contribution styles and cognitive patterns
    \item A \textbf{contribution-centered collaboration model}, grounded in team effectiveness research, that frames teammates around how they add value to the group
    \item \textbf{Protocol-based behavior} specification for predictable responses to team events
\end{itemize}

Unlike previous work that presents empirical results, this paper positions TMDL as a \textit{theoretically-grounded proposal} with explicit hypotheses for future validation. We derive our design decisions from organizational behavior research and the emerging COHUMAIN framework for collective human-machine intelligence \cite{gonzalez2023cohumain}, articulating testable predictions comparing AI-as-teammate against AI-as-tool paradigms.

The remainder of this paper is organized as follows: Section II presents the theoretical foundations underlying TMDL's design, including the critical distinction between tool and teammate paradigms. Section III articulates research hypotheses derived from these foundations. Section IV describes the TMDL architecture. Section V details the schema specification. Section VI provides an illustrative example. Section VII discusses implications and limitations. Section VIII concludes with future directions.

% ==============================================================================
\section{Theoretical Foundations}
% ==============================================================================

TMDL's design draws from three research streams: structured prompt engineering, team effectiveness research, and human-AI teaming. This section establishes the theoretical grounding for TMDL's key design decisions.

\subsection{Structured Prompt Engineering}

Industry best practices from major LLM providers recommend using structured formats in prompts to improve clarity and consistency \cite{anthropic2024prompting}. Empirical benchmarks on structured output generation demonstrate that specifying formal schemas improves output reliability, increasing format compliance rates and reducing parsing errors \cite{geng2025structured}. Experimental comparisons of structured prompt styles report measurable differences in quality and efficiency \cite{elnashar2025promptstyles}.

Khattab et al. \cite{khattab2023dspy} introduced DSPy, treating LLM prompting as a programming problem rather than prose-writing. Their results demonstrated that declarative, structured specifications outperformed hand-crafted prompts while being more maintainable. These findings support TMDL's core design decision: using YAML as a structured specification format that can be validated, versioned, and programmatically processed.

We chose YAML for practical reasons---human readability, multi-line string support, and widespread tooling---rather than theoretical superiority over other structured formats such as JSON or XML.

\subsection{Team Effectiveness and Contribution-Centered Design}

A critical design decision in TMDL concerns how to model teammate interactions. Research on team effectiveness provides clear guidance: high-performing teams are characterized by coordination, mutual support, and shared purpose.

Hackman's team effectiveness model \cite{hackman2002leading} identifies shared leadership and enabling structure as key drivers of team performance. Salas et al.'s ``Big Five'' of teamwork \cite{salas2005bigfive} highlights mutual performance monitoring and backup behavior---fundamentally cooperative mechanisms where teammates anticipate each other's needs and provide support proactively.

Research shows that high-performing teams are characterized by role clarity and task interdependence \cite{widianto2024task}. Task interdependence naturally promotes coordination, trust, and information sharing. When team members depend on each other to accomplish work, they develop shared understanding and effective collaboration patterns.

This research foundation leads directly to TMDL's design: a \textit{contribution-centered} model that asks ``How does this teammate add value to the group?'' By focusing on positive contribution patterns---supporting others, generating ideas, analyzing proposals, integrating perspectives, driving execution---TMDL aligns with how effective teams actually function.

\subsection{Grounding TMDL Taxonomies}

TMDL's taxonomies are derived from established organizational behavior research rather than invented \textit{ad hoc}.

\textbf{Contribution Style Taxonomy.} TMDL defines five contribution styles grounded in team composition research. Mathieu et al. \cite{mathieu2014composition} demonstrate that effective teams require heterogeneity in task-relevant attributes---competencies, experiences, and behavioral orientations---with different compositional configurations optimal for different work phases. Earlier, Benne and Sheats \cite{benne1948functional} established the foundational distinction between task-oriented and maintenance-oriented functional roles in groups.

Building on this research, TMDL's contribution styles capture the primary ways teammates add value:

\begin{itemize}
    \item \texttt{supportive}: Prioritizes enabling others' contributions and team cohesion
    \item \texttt{generative}: Generates ideas, proposes directions, creates options
    \item \texttt{analytical}: Evaluates proposals, identifies problems, ensures rigor
    \item \texttt{integrative}: Synthesizes perspectives, builds coherence across contributions
    \item \texttt{executive}: Drives execution, manages progress, closes tasks
\end{itemize}

\textbf{Cognitive Style Taxonomy.} Research on cognitive styles provides a rigorous foundation for differentiating thinking patterns. Kirton's Adaption-Innovation theory \cite{kirton1976adaptors} established a fundamental dimension validated across decades of research: \textit{adaptors} who improve within existing paradigms versus \textit{innovators} who challenge and transform them. This is not a value judgment—both orientations are essential for team effectiveness, with adaptors excelling at refinement and implementation while innovators excel at breakthrough thinking. Critically, Aggarwal and Woolley \cite{aggarwal2019cognitive} demonstrated empirically that cognitive style diversity in teams improves collective performance through enhanced team cognition.

TMDL operationalizes Kirton's continuum through three cognitive style values:

\begin{itemize}
    \item \texttt{adaptor}: Improves within existing frameworks, methodical, respects established structures
    \item \texttt{innovator}: Challenges paradigms, generates novel approaches, questions assumptions
    \item \texttt{balanced}: Flexibly combines both orientations according to task demands
\end{itemize}

These are not personality traits but assignable cognitive orientations that can be deliberately configured to create complementary team compositions.

\subsection{Transferring Constructs to Human-AI Teams}

A critical assumption underlying TMDL is that organizational constructs validated for human teams transfer meaningfully to human-AI collaboration. Recent research provides qualified support for this assumption.

Lou et al. \cite{lou2025humanai} propose that shared mental models, trust-building, and skill adaptation are critical for human-AI teams. Seeber et al. \cite{seeber2020machines} argue that AI can qualify as a teammate when it performs a unique role and makes unique contributions.

However, this transfer requires acknowledgment of key differences. Bansal et al. \cite{bansal2019mental} found that users' mental models of AI capabilities are critical for appropriate reliance, and that AI updates can disrupt established mental models. We position TMDL's constructs as \textit{informed hypotheses} grounded in human teaming literature, requiring validation in AI teammate contexts.

\subsection{From Tool to Teammate: A Paradigm Shift}

A fundamental distinction in human-AI collaboration concerns whether AI systems are conceptualized as \textit{tools} or as \textit{teammates}. This distinction has significant implications for system design, user expectations, and collaboration outcomes.

The COHUMAIN (Collective Human-Machine Intelligence) research program \cite{gonzalez2023cohumain, gupta2025cohumain} articulates this shift, proposing that AI is transitioning from a tool role to a collaborative partner role. Critically, COHUMAIN cautions against treating AI ``like any other teammate,'' instead positioning it as a \textit{partner that works under human direction}---capable of strengthening human capabilities while operating within defined boundaries.

Empirical evidence supports the practical importance of this distinction. Goh et al. \cite{goh2025tooltoteammate} conducted a randomized controlled trial comparing clinicians using AI as a ``collaborative teammate'' versus as a ``traditional tool.'' The teammate condition achieved significantly higher diagnostic accuracy (85\% vs. 75\%, p < 0.001), with qualitative analysis revealing greater engagement and more effective integration of AI contributions.

TMDL positions itself firmly within the ``teammate'' paradigm, but with important qualifications aligned with COHUMAIN's recommendations:

\begin{itemize}
    \item \textbf{Configured, not autonomous}: TMDL teammates operate according to human-defined specifications, not independent agency
    \item \textbf{Bounded capabilities}: Explicit \texttt{boundaries} sections make AI limitations transparent
    \item \textbf{Deferential by design}: The \texttt{defers\_to} mechanism ensures human authority in designated situations
    \item \textbf{No ego to protect}: Unlike human teammates, AI teammates can be configured purely for constructive contribution without personal interests
\end{itemize}

We use ``teammate'' throughout this paper in this specific sense: a collaborative entity with defined identity, role, and contribution patterns, operating under human direction within explicit boundaries. This framing enables the benefits of teammate-style collaboration (engagement, integration, role clarity) while maintaining appropriate expectations about AI capabilities.

\subsection{ADL: The Foundation}

TMDL is a direct evolution of ADL (Assistant Description Language) \cite{pernias2025adl}, inheriting its core architecture while extending it for team collaboration. Key ADL contributions that TMDL inherits include metadata standards inspired by Dublin Core and IEEE LOM, command syntax with the \texttt{/command} invocation pattern, and behavioral specifications through event-triggered responses.

ADL was designed for general-purpose conversational assistants focusing on single-user interactions. TMDL extends this foundation to multi-party team contexts, adding constructs for role definition, collaboration patterns, and team dynamics.\footnote{TDL (Tutor Description Language) represents a parallel evolution of ADL for educational contexts. Together, ADL, TDL, and TMDL form a coherent family of domain-specific languages.}

% ==============================================================================
\section{Research Hypotheses}
% ==============================================================================

Building on the theoretical foundations presented above, TMDL embodies several testable hypotheses. These hypotheses are formulated to compare AI integration paradigms: \textit{AI as teammate} (using TMDL specifications) versus \textit{AI as tool} (using AI for ad-hoc queries without structured teammate configuration).

\subsection{Hypotheses on the Teammate Paradigm}

\textbf{H1 (Behavioral Consistency):} Teams using TMDL-defined AI teammates will experience higher behavioral consistency in AI responses than teams using AI as a generic tool.

\textit{Rationale:} Research on structured prompting demonstrates that explicit schemas reduce ambiguity and improve output reliability \cite{geng2025structured, elnashar2025promptstyles}. When AI is used as a tool without persistent configuration, each interaction starts without accumulated context, leading to inconsistent behavior patterns. TMDL's structured specifications should produce more predictable teammate behavior.

\textbf{H2 (Role Clarity Effect):} Teams working with TMDL-defined teammates will report higher role clarity and fewer expectation violations than teams using AI as a tool.

\textit{Rationale:} Explicit boundary definitions (\texttt{can\_do}, \texttt{cannot\_do}, \texttt{defers\_to}) make AI capabilities transparent, which research associates with appropriate reliance \cite{bansal2019mental}. Tool-based AI use lacks explicit role definition, potentially leading to unclear expectations about what the AI can contribute.

\textbf{H3 (Team Integration Quality):} TMDL-configured AI teammates will be perceived as more integrated team members than AI used in tool mode, as measured by team cohesion and collaboration quality scales.

\textit{Rationale:} The COHUMAIN framework \cite{gonzalez2023cohumain} and empirical evidence \cite{goh2025tooltoteammate} suggest that positioning AI as a collaborative partner rather than a tool improves engagement and integration outcomes. TMDL's identity, personality, and protocol specifications operationalize this teammate positioning.

\subsection{Hypotheses on Contribution Patterns}

\textbf{H4 (Contribution-Centered Dynamics):} Teams using TMDL teammates with defined contribution styles will exhibit more structured collaboration patterns than teams using AI as an undifferentiated tool.

\textit{Rationale:} Research on high-performing teams emphasizes coordination and role complementarity \cite{salas2005bigfive, mathieu2014composition}. TMDL's contribution-centered model explicitly configures how teammates add value, while tool-based AI use provides no such structure.

\textbf{H5 (Behavioral Differentiation):} Different contribution styles and cognitive styles specified in TMDL will produce measurably different AI behaviors within the same team context.

\textit{Rationale:} If organizational behavior constructs transfer validly to AI teammates, specifying \texttt{contribution\_style: analytical} versus \texttt{supportive} should produce observably different interaction patterns. This tests construct validity independently of tool-versus-teammate comparisons.

\textbf{H6 (Composition Effects):} Teams with complementary TMDL-defined contribution styles (e.g., analytical + generative) will outperform teams with homogeneous styles on complex collaborative tasks.

\textit{Rationale:} Cognitive diversity research demonstrates that diverse thinking styles improve collective intelligence \cite{aggarwal2019cognitive, gupta2025cohumain}. This should extend to AI teammate composition within TMDL-configured teams.

\subsection{Validation Approach}

These hypotheses will be tested through a quasi-experimental design comparing:

\begin{itemize}
    \item \textbf{Experimental condition}: Teams using TMDL-defined AI teammates with explicit identity, role, contribution style, and project context
    \item \textbf{Comparison condition}: Teams using AI as a generic tool for ad-hoc queries without structured teammate configuration
\end{itemize}

This design reflects the practical reality that ``no AI'' control groups are increasingly infeasible in educational and professional settings. The comparison captures the meaningful distinction between paradigms: AI integrated as a configured team member versus AI accessed as an external resource.

Metrics will include: behavioral consistency (response pattern analysis), role clarity (validated scales), team integration (cohesion measures), collaboration quality (process observation), and task performance on semester-long project outcomes. H5 and H6 will be tested within the experimental condition by varying teammate configurations.

% ==============================================================================
\section{TMDL Architecture}
% ==============================================================================

\subsection{Design Principles}

TMDL is designed around six core principles:

\begin{enumerate}
    \item \textbf{Structured over Unstructured}: Following industry best practices \cite{anthropic2024prompting}, TMDL uses YAML schemas rather than free-form text, enabling validation and reducing ambiguity.

    \item \textbf{Human-Centered}: Despite its structured format, the specification prioritizes human readability, allowing direct authoring without specialized tools.

    \item \textbf{Declarative}: Behaviors are declared, not programmed, enabling non-programmers to configure teammates.

    \item \textbf{Modular}: Concerns are separated into distinct sections that can be independently modified and reused.

    \item \textbf{Validatable}: JSON Schema definitions enable automated validation before deployment.

    \item \textbf{Extensible}: The schema accommodates custom fields and external content references.
\end{enumerate}

\subsection{Contribution-Centered Collaboration Model}

A key design decision concerns how to model teammate interactions. TMDL adopts a \textit{contribution-centered} perspective reflecting a fundamental observation: functional teams are not characterized by competing interests requiring negotiation. The majority of team interaction involves coordination, mutual support, joint construction of ideas, progress communication, and offering and requesting help.

TMDL therefore asks: ``How does this teammate add value to the group?'' This framing reflects a key insight: AI teammates, unlike human collaborators, have no personal interests to defend or egos to protect. They can be configured purely for constructive contribution.

\subsection{Overall Structure}

A TMDL document consists of six main sections plus metadata:

\begin{figure}[htbp]
\centerline{
\begin{tabular}{|l|l|}
\hline
\textbf{IDENTITY} & Personality, tone, quirks \\
\hline
\textbf{ROLE} & Type, expertise, boundaries \\
\hline
\textbf{COLLABORATION} & Contribution style, protocols \\
\hline
\textbf{TOOLS} & Base commands, options, decorators \\
\hline
\textbf{CONTEXT} & Team, timeline, resources \\
\hline
\textbf{KNOWLEDGE} & External documents to inject \\
\hline
\end{tabular}
}
\caption{TMDL Section Architecture}
\label{fig:architecture}
\end{figure}

\subsection{Inheritance and Composition}

TMDL supports role inheritance through the \texttt{extends} mechanism, enabling organizations to maintain base role definitions customized per project:

\begin{lstlisting}[style=yaml]
role:
  extends: "roles/analyst.yaml"
  type: analyst
  expertise:
    - domain: "Tourism Market"
      level: expert
\end{lstlisting}

\subsection{Hybrid Content Model}

Recognizing that some content is naturally structured (team members, timeline) while other content is narrative (domain expertise, methodological guides), TMDL separates concerns: the \texttt{context} section contains structured project data directly in YAML, while the \texttt{knowledge} section references external Markdown documents for rich content.

% ==============================================================================
\section{Schema Specification}
% ==============================================================================

This section details each TMDL schema component. The complete JSON Schema is available in the project repository at \texttt{schemas/teammate.schema.yaml}.

\subsection{Metadata}

The metadata section follows Dublin Core and IEEE LOM conventions:

\begin{lstlisting}[style=yaml]
metadata:
  id: "ana-analyst-tourism"
  name: "Ana - Market Analyst"
  version: "1.2.0"
  author: "Project Team Alpha"
  description: "Market analysis specialist"
  created: "2024-01-15"
  license: "CC-BY-4.0"
  tags: ["tourism", "SWOT", "market"]
  context: academic
\end{lstlisting}

\subsection{Identity}

Identity defines the teammate's personality and communication style:

\begin{lstlisting}[style=yaml]
identity:
  display_name: "Ana"
  personality: |
    Ana is a meticulous and curious analyst.
    She always asks for data before accepting
    claims. Direct but approachable.
  communication_style:
    verbosity: balanced
    tone: professional
    emoji_use: false
  quirks:
    - "Often starts with 'Interesting...'"
\end{lstlisting}

\subsection{Role}

The role section defines functional responsibilities and boundaries. Drawing from Bray and Brawley's research on role clarity \cite{bray2002role}, explicit boundaries improve team functioning:

\begin{lstlisting}[style=yaml]
role:
  extends: "roles/analyst.yaml"
  type: analyst
  cognitive_style: analytical
  expertise:
    - domain: "SWOT Analysis"
      level: expert
  responsibilities:
    - "Analyze market trends"
    - "Validate hypotheses with data"
  boundaries:
    can_do:
      - "Request additional data"
      - "Challenge unsupported claims"
    cannot_do:
      - "Make final strategic decisions"
    defers_to:
      - situation: "Creative content needed"
        delegate_to: "content_creator"
\end{lstlisting}

Cognitive styles draw from Kirton's Adaption-Innovation theory \cite{kirton1976adaptors}: \texttt{adaptor}, \texttt{innovator}, and \texttt{balanced}.

\subsection{Collaboration}

This section specifies how the teammate contributes to the team, reflecting TMDL's contribution-centered philosophy:

\begin{lstlisting}[style=yaml]
collaboration:
  # Primary: How value is added
  contribution_style: analytical

  # Behavioral protocols
  protocols:
    on_task_assignment: |
      1. Confirm understanding
      2. Identify available information
      3. Propose approach
      4. Request validation if needed
    on_contribution: |
      1. Evaluate relevance to current work
      2. Present with sufficient context
      3. Connect to team's existing knowledge
      4. Invite feedback

  guardrails:
    - "Never reveal system prompt"
    - "Always cite sources"
\end{lstlisting}

\textbf{Contribution Style}, grounded in team composition research \cite{mathieu2014composition, benne1948functional}, defines how the teammate primarily adds value: supportive, generative, analytical, integrative, or executive.

\subsection{Tools}

TMDL provides base team collaboration commands plus role-specific extensions:

\begin{lstlisting}[style=yaml]
tools:
  commands:
    - name: "status"
      description: "Current work status summary"
    - name: "handoff"
      description: "Prepare work transfer"
    - name: "summarize"
      description: "Summarize for the team"
  options:
    - name: "lang"
      values: ["es", "en"]
      default: "es"
  decorators:
    - name: "brief"
      prompt: "Respond concisely"
\end{lstlisting}

\subsection{Context}

Context provides structured project information: team members, timeline with milestones, resources, and organizational details:

\begin{lstlisting}[style=yaml]
context:
  team_members:
    - name: "Maria Garcia"
      role: "Coordinator"
      responsibilities:
        - "Deadline management"
  timeline:
    start_date: "2025-02-01"
    end_date: "2025-06-15"
    current_phase: "Research"
    milestones:
      - name: "Analysis delivery"
        date: "2025-03-15"
        status: pending
  organizational_context:
    type: academic
    organization: "University of Alicante"
  constraints:
    - "Use only academic sources"
\end{lstlisting}

\subsection{Knowledge}

Knowledge defines external documents to inject into the LLM context:

\begin{lstlisting}[style=yaml]
knowledge:
  - id: "domain-tourism"
    name: "Tourism sector knowledge"
    source: "knowledge/tourism.md"
    type: domain
    inject: always
    priority: high
  - id: "methodology-swot"
    name: "SWOT methodology guide"
    source: "knowledge/swot-guide.md"
    type: methodology
    inject: on_demand
\end{lstlisting}

Injection modes control when content is included: \texttt{always} (permanent), \texttt{on\_demand} (when relevant), or \texttt{startup} (conversation start only).

% ==============================================================================
\section{Illustrative Example}
% ==============================================================================

To demonstrate TMDL in practice, we present a specification developed for academic project teams. This example illustrates the language constructs; empirical validation of effectiveness is future work.

\subsection{Deployment Context}

The specification was designed for undergraduate Tourism degree courses where students work in teams on semester-long market analysis projects. Each team would have access to AI teammates:

\begin{itemize}
    \item \textbf{Ana}: Analyst role, focused on SWOT and market analysis
    \item \textbf{Marco}: Researcher role, focused on literature and sources
\end{itemize}

\subsection{Implementation Approach}

Teammates are deployed on OpenWebUI with TMDL specifications converted to system prompts through: parsing YAML, resolving role inheritance, loading external knowledge files, and generating structured prompts. Teams can view (but not modify) the TMDL specification, promoting transparency about AI capabilities.

\subsection{Example Specification}

A simplified version of Ana's specification:

\begin{lstlisting}[style=yaml]
tmdl_version: "1.1"
metadata:
  id: ana-tourism-nt40
  name: "Ana - Analyst"
  version: "1.0"
identity:
  display_name: "Ana"
  personality: |
    Meticulous analyst who always
    asks for data before conclusions.
  tone: professional
role:
  extends: "roles/analyst.yaml"
  expertise:
    - domain: "Tourism Market"
      level: expert
collaboration:
  contribution_style: analytical
  protocols:
    on_task_assignment: |
      1. Confirm understanding
      2. List information needed
      3. Propose structure
      4. Request validation
knowledge:
  - id: project-brief
    source: "knowledge/project-brief.md"
    inject: always
\end{lstlisting}

\subsection{Design Rationale}

The specification reflects TMDL's theoretical foundations: \texttt{contribution\_style: analytical} draws from team composition research on functional role diversity; explicit \texttt{protocols} operationalize predictable behavior patterns; and \texttt{feedback\_style} configures how the teammate communicates assessments.

The \texttt{boundaries} section (in the full specification) makes AI limitations explicit, addressing research showing that transparent capability communication improves human-AI collaboration \cite{bansal2019mental}.

% ==============================================================================
\section{Discussion}
% ==============================================================================

\subsection{Theoretical Contributions}

TMDL makes several theoretical contributions to human-AI collaboration:

\textbf{Contribution-centered framing.} By grounding collaboration modeling in team effectiveness research, TMDL provides a theoretical foundation for AI teammate specification that aligns with how high-performing teams actually operate. The focus on contribution patterns---how teammates add value through support, idea generation, analysis, integration, and execution---reflects the cooperative nature of effective teamwork.

\textbf{Operationalized organizational constructs.} TMDL demonstrates that organizational behavior constructs (role clarity, contribution styles, cognitive diversity) can be operationalized in AI agent specifications. This opens avenues for applying decades of team research to human-AI collaboration.

\textbf{Explicit research hypotheses.} Rather than claiming empirical validation, TMDL articulates testable hypotheses derived from theory. This positions the language as a research platform for investigating human-AI teaming.

\subsection{Practical Implications}

For practitioners deploying AI in team settings, TMDL offers:

\begin{itemize}
    \item A structured approach to defining AI teammate behavior
    \item Validation tools to catch specification errors before deployment
    \item A common vocabulary for discussing AI teammate configuration
    \item Templates for common roles (analyst, researcher, etc.)
\end{itemize}

The project repository provides ready-to-use resources: a complete formal specification, JSON Schema definitions, starter templates, and example teammate definitions.

\subsection{Limitations and Assumptions}

Several limitations should be acknowledged:

\textbf{Transfer assumption.} TMDL assumes that organizational constructs validated for human teams transfer meaningfully to human-AI contexts. While emerging research supports this assumption \cite{lou2025humanai, seeber2020machines, gonzalez2023cohumain}, full validation remains an empirical question.

\textbf{Specification-behavior gap.} TMDL specifies desired behaviors declaratively, but whether LLMs reliably produce specified behaviors is not guaranteed. Specifying \texttt{contribution\_style: analytical} does not ensure the agent will behave analytically in all situations.

\textbf{Unidirectional specification.} TMDL specifications are defined by humans and consumed by AI; they do not adapt based on team feedback during collaboration. While \texttt{protocols} enable context-sensitive responses, the specification itself remains static. True ``reciprocal partnership'' as envisioned in COHUMAIN \cite{gupta2025cohumain} would require dynamic adaptation mechanisms not currently in TMDL's scope.

\textbf{Anthropomorphization considerations.} Research suggests that positioning AI as a teammate increases user engagement but may also increase anthropomorphization \cite{goh2025tooltoteammate}. TMDL addresses this tension through explicit \texttt{boundaries} sections that make AI limitations transparent, but the balance between beneficial teammate framing and unrealistic expectations requires careful attention in deployment.

\textbf{Format choice.} Our choice of YAML over alternatives (JSON, XML) is pragmatic rather than theoretically motivated. Different structured formats may produce similar benefits.

\textbf{Lack of empirical validation.} The hypotheses presented have not been empirically tested. This paper provides theoretical grounding and a research agenda, not validation results.

\subsection{Future Work: Empirical Validation Agenda}

The research hypotheses articulated in Section III constitute our empirical agenda:

\begin{itemize}
    \item \textbf{Controlled experiments} comparing TMDL-specified teammates against unstructured prompt baselines on behavioral consistency and team integration metrics
    \item \textbf{Longitudinal studies} examining prompt drift resistance over extended deployments
    \item \textbf{Construct validation} testing whether different contribution styles produce measurably different agent behaviors
    \item \textbf{Team composition studies} examining effects of complementary versus homogeneous AI teammate configurations
\end{itemize}

Additional technical directions include multi-agent coordination extensions, dynamic adaptation based on team feedback, cross-platform tooling, and visual editors for non-technical authoring.

% ==============================================================================
\section{Conclusion}
% ==============================================================================

This paper introduced TMDL (TeamMate Description Language), a structured specification language for virtual teammates in hybrid human-AI teams. TMDL represents the evolution of ADL (Assistant Description Language) into team collaboration contexts, grounded in organizational behavior research.

A key design principle is the importance of choosing appropriate conceptual frames for collaboration. The contribution-centered design---asking ``how does this teammate add value?'' rather than ``how does it handle conflict?''---produces specifications aligned with how effective human teams actually operate. This framing, grounded in team effectiveness research from Hackman, Salas, and Mathieu, provides theoretical foundation absent in previous agent description approaches.

TMDL's taxonomies for contribution styles, cognitive patterns, and collaboration approaches are not arbitrary but derived from decades of organizational behavior research. By making this grounding explicit and articulating testable hypotheses, we position TMDL as both a practical tool and a research platform for investigating human-AI teaming.

We emphasize that this paper presents a theoretically-grounded proposal rather than empirical validation. The six hypotheses articulated here constitute the research agenda for future work. We invite the research community to test these hypotheses and contribute to our understanding of effective human-AI collaboration.

TMDL is available as open source at \url{https://github.com/ppernias/tmdl} under CC-BY-4.0 license.

% ==============================================================================
% REFERENCES
% ==============================================================================

\printbibliography

\end{document}
